\chapter{Introduction to Modules}
\section{\texttt{Defination of Module}}
\subsection{Left Module:}\label{sec:left-module}\index{Left Module}
 Let $ R $ be a ring with identity and $ M $ be an abelian group with addition. We say $ M $ is a left $R-$module if there exists a mapping\footnote{often called as scaler multiplication.}
\begin{equation*}
R \times M \rightarrow M
\end{equation*}	
defined by
\begin{equation*}
(a \ , \ x) \rightarrow ax
\end{equation*}\marginnote[-2em]{$\forall \ a \in \ R$ and $x$ $\in \ M$}
satisfying following properties :
\begin{align}
(a+b) x =& ax + bx \\ a(x+y) =& ax + ay \\ (ab)x =& a(bx) \\ 1x =& x
\end{align}\marginnote[-5em]{$\forall$ $ a\ , \ b \in R$ and $x \ , y \ \in M$}
and denoted by $_{R}M$
\subsection{Right Module:}\label{sec:right-module}\index{Right Module}
Let $ R $ be a ring with identity and $ M $ be an abelian group with addition. We say $ M $ is a right $R-$module if there exists a mapping
\begin{equation*}
M \times R \rightarrow M
\end{equation*}	
defined by
\begin{equation*}
(x \ ,\ a) \rightarrow xa
\end{equation*}
\marginnote[-1em]{$ \forall \ a$ $\in$ $R$ and $x$ $\in$ $M$} satisfying following properties\marginnote[+4em]{$\forall$ $ a\ , \ b \in R$ and $x \ , y \ \in M$  } :
\begin{align}
x (a+b)  &= xa + xb \\ (x+y) a &= xa + ya \\ x (ab) &= (xa)b \\ x1 &= x
\end{align}
and denoted by $M_{R}$.

\subsection{Examples :}
\begin{enumerate}
\item Let $V$ be a vector space over a field $F$ then $V$ is a left as well as right $F-$Module.\newline \bigskip
\item Let $G$ be any abelian group under addition , then $G$ is a $\mathbb{Z-}$Module where $\mathbb{Z}$ is set of integers.\newline \bigskip
\item Let $R$ be ring and $M= R[x]$ \marginnote{Suppose ring $R$ is a field then $R-$Module $R[x]$ is a vector space over field $R$.} where $R[x]$ is a group of all polynomials with coefficents in $R$ then $M$ is a left as well as a right $R-$Module with scaler multiplication being usual multiplication.\newline \bigskip
\item Let $M$ be collection of all $m \times n $ matrices over ring $R$ , then $M$ is left $R-$Module where scaler multiplication being usual multiplication of a scaler to a matrix. \newline
\bigskip
In particular, if $M$ is a set of $1 \times n$ matrices over $R$ or $M = R^n$(set of $n-$tuples) then $R^n$ is a left $R-$module.
\end{enumerate}
\bigskip
\begin{remark}
	Let $R$ be a commutative ring then every left $R-$module can be transformed to right $R-$module and vice-versa.
	\end{remark}
\begin{proof} Let $M$ be left $R-$module and $R$ be a commutative ring. \newline
	so, $\exists$ a mapping
\begin{equation*}
R \times M \rightarrow M
\end{equation*}	
defined by
\begin{equation*}
(a \ , \ x) \rightarrow ax
\end{equation*}
for each $a$ $\in$ $R$ and $x$ $\in$ $M$ satisfying following properties :
\marginnote{$\forall$ $ a\ , \ b \in R$ and $x \ , y \ \in M$  }
\begin{eqnarray*}
(a+b) x &=& ax + bx \\ a(x+y) &=& ax + ay \\ (ab)x &=& a(bx) \\ 1x &=& x
\end{eqnarray*}
$\because \ R$ is a commutative ring.\newline
Now, Define an another mapping
 \begin{equation*}
 M \times R \rightarrow M
 \end{equation*}	
 defined by
 \begin{equation*}
 (x \ ,\ a) \rightarrow x*a \ = \ ax
 \end{equation*}
To check $M$ is a right $R-$Module , we need to verify properties number \ref{eq:eq1.5}-\eqref{eq1.8}
 \begin{description}
 	\item[(i) Distribuitive Law]
 	\begin{align*}
 		x*(a+b) &= (a+b)x \\ &= ax + bx \\ &= (x*a) + (x*b)
 	\end{align*}
 	\item[(ii) Distributive Law] \begin{align*}
 		(x+y)*a &= a(x+y) \\ &= ax + ay \\ &= (x*a) + (y*a)
 	      \end{align*}
       \item[(iii)]  \begin{align*}
       x*(ab) &= (ab)x \\ &= (ba)x \\ &= b(ax) \\ &= (ax)*b
       \end{align*}
   \item[(iv)] \begin{align*}
   x*1 &= 1x \\  &= x
   \end{align*}
Thus, $_{R}M$ is transformed to $M_R$.\newline
Similarly, Converse statement can be verified.
\end{description}
\end{proof} \bigskip
\begin{remark}
	Let $S$ be a subring of ring $R$ then $_{S}M$ exists \marginnote{by existance means $M$ is a valid left module over mentioned ring or subring. i.e. satisfying those four properties.}  only if $_{R}M$ exists.
\end{remark} \bigskip
\begin{remark}
	Same Abelian group\marginnote{For Instance,  The field $\mathbb{R}$ is $\mathbb{R}-$module,$\mathbb{Q}-$module and $\mathbb{Z}-$module.} can have the structure of a Module for a number of different rings.
\end{remark} \bigskip
\begin{remark}
Let $I$ be left ideal of $R$ then quotient ring $\bigslant{R}{I}$ is a left $R$-module.\marginnote{Here scaler multiplication is \newline \begin{equation*}
	R \times \bigslant{R}{I} \rightarrowtail \bigslant{R}{I}
	\end{equation*}
defined as
\begin{equation*}
(a \ , \ x+I) \rightarrowtail ax+I
\end{equation*}

$\forall \ a \in R$ and $\forall \ x+I \in \bigslant{R}{I}$}

\end{remark}
\begin{proof}[verification:]
	`left to reader'\newline \bigskip
	\textbf{Hint:} you need to verify those four properties: \eqref{eq1.1}-\eqref{eq1.4}
\end{proof}
\bigskip
\begin{thm}{(Elementry Properties:)}\newline
	Let $M$ be a left $R$-module . Suppose $0_m \ \text{and} \ 0_r$ denotes additive identities of $M$ and $R$ respectively. Then,  for each $x \ \in M$ and $r \ \in R$ \newline
\begin{description}
\item (i)
	\begin{equation*}
	0_m = 0_r\ x = r\ 0_m
	\end{equation*}
\item (ii)
	\begin{equation*}
	r(-x) =  (-r)x = -rx
	\end{equation*}
\end{description}
\end{thm}
\begin{proof}
	\begin{description}
		\item[(i)]
As $0_m$ is the additive identity of $M$. so, $0_m = 0_m + 0_m$\newline
Consider \marginnote{$ \because \ (r\ ,\ 0_m)\rightarrowtail r\ 0_m \in M$\newline so, $r\ 0_m = r\ 0_m + 0_m$}
 $ r(0_m + 0_m) = r\ 0_m  = r\ 0_m +0_m $\newline
but, $r(0_m + 0_m) = (r\ 0_m) +(r\ 0_m)$ \marginnote{
	$ \because \ M$ is a left $R$-module.\newline (using distribuitive property)}\newline
so, we have \begin{equation*}
r\ 0_m +r\ 0_m = r\ 0_m + 0_m
\end{equation*}
as ($M,+$) is an abelian group so left and right cancellation law holds.
\begin{eqnarray*}
\cancel{r\ 0_m} +r\ 0_m &=& \cancel{r\ 0_m} + 0_m \\ r\ 0_m &=& 0_m
\end{eqnarray*}
a similiar argument can be used to prove $0_m = 0_r\ x$.
\item[(ii)]
as $M$ is a left $R$-module so $(r \ , x )\ \rightarrowtail rx \in M$ \newline
Now, Consider $(-r)x + rx$
\begin{eqnarray*}
\text{using distribuitive law} \\ (-r)x + rx &=& (-r + r)x \\  &=& 0_r \ x \\ &=& 0_m
\end{eqnarray*}
i.e. $(-r)x$ is additive inverse of  $(rx)$ but additive inverse of $(rx)$ is $-rx$ and it is unique for an abelian group($M$ here)
\begin{equation*}
\therefore \ (-r)x = -rx
\end{equation*}
a similar argument can be used to prove that $r(-x)  = -r x$.
\end{description}
\end{proof}
\bigskip
\begin{defn}[Ring Homomorphism]
	Let\marginnote{often called as ring homo} $R$ and $S$ be two rings with identities $1_r \ , 1_s$ respectively then a map(say $f$)
	\begin{equation*}
	f : R \rightarrow S
	\end{equation*}
	is said to be a ring homomorphism or ring linear map if for every $a \ ,\ b \in R$ following properties holds\marginnote[-2.55em]{if $R$ = $S$ then we call ring homo as ring endomorphism. For instance , let $f$ be ring homo from $R$ to $R$ . we say $f$ is endomorphism of $R$ and denoted by $End \ R$}
	\begin{description}
		\item[(i) \centering{Preserves Addition}]\begin{equation*}
		(a+b)f = (a)f +(b)f
		\end{equation*}
		\item[(ii) \centering{Preservers Multiplication} ]\begin{equation*}
		(ab)f = (a)f . (b)f
		\end{equation*}
		\item[(iii) \centering{ Maps identity to identity}]\begin{equation*}
		(1_r)f = 1_s
		\end{equation*}
	\end{description}
\end{defn}
\begin{remark}
	Such a mapping need not to be bijective. if it is bijective then we say it is a ring isomorphism or rings are isomorphic.
\end{remark}
\bigskip
\begin{thm}
	Let\marginnote{\[ M \ \text{is a right R-module} \] \[\centering\Updownarrow \] \[\newline\exists \ f: R \ \xrightarrow[\text{Homo}]{\text{Ring}} End \ M \] } $R$ be a ring and $M$ be any abelian group with addition. then $M$ is a right $R$-module if and only if there exists a map which is ring homomorphism from $R$ to $End \ M$
\end{thm}
\begin{proof}
\begin{description}
	\item[(Forward Part)\newline]
	Let us suppose that $ M $ is a right $ R $-module.\newline
	\textbf{Claim:} there exists a map which is ring homomorphism from $R$ to $End \ M$\newline
   $ \because \ M  $ is a left $ R $-module , so there exist a map\newline
   \[ f: M \times R \rightarrowtail M  \]
   defined by
   \[ (x \ , a) \rightarrowtail ax \]
   satisfying following properties:
    \begin{align*} (x+y)a &= (x)a+(y)a \\
     x(a+b)\mathnote{ $$\forall \ x , y \ \in M \ \&  \ a ,b \in R$$ } &= xa + xb \\
     x(ab) &= (xa)b \\
     x 1 &= x
     \end{align*}
   for each $ a \in R $ , define a map(say $\phi_a $)
   \[ \phi_a: M\rightarrowtail\ M \]
   such that for each $x\in  M$
   \[ (x )\phi_a = xa \in \ M \]
   Now, we'll show that $ \phi_a \in \ End \ M $\newline
   Let $ x , y \in M $ \newline Consider  $  (x+y)\phi_a$
   \begin{align*}
   	&= (x+y)a  \\ &= xa + ya  \\ &= (x)\phi_a + (y)\phi_a
   \end{align*}\marginnote[-5.5em]{using defination of $\phi_a$} \marginnote[-4em]{using \eqref{eq1.9}}
   so, $\phi_a$ preserves addition and is a group homo from $M$ to $M$.
   \newline i.e. $\phi_a \in End \ M$ \newline
   Now, we can define a map (say $f$)
   \begin{align*}
   &f : R \rightarrowtail End \ M \\ \text{defined as} \\&(a)f \rightarrowtail \phi_a \mathnote{$\forall a \in R$ and $\phi_a \in End \ M$}
   \end{align*}
Now, We'll show that $ f $ is a ring homomorphism.\newline
\begin{description}
	\item[(A)]
	\begin{align*}
	(a+b)f &= \phi_{a+b}
	\mathnote{\[ \text{for each}\  x \in M \ \text{we have,}\]
		\begin{align*} (x)\phi_{a+b} &= x(a+b) \\ &=xa+xb &=(x)\phi_a +(y)\phi_b \\ \therefore \ \phi_{a+b} &= \phi_a +\phi_b \end{align*}}
		  \\  &= \phi_a +\phi_b \\&= (a)f +(b)f
	\end{align*}
    \item[(B)] \begin{align*}
    (ab)f &= \phi_{ab}\mathnote{\[ \text{for each}\  x \in M \ \text{we have,}\] \begin{align*} (x)\phi_{ab} &= x(ab) \\&=(xa)b \ =\ (xa)\phi_b \\&= (x)\phi_a \centerdot\phi_b \end{align*}\[ \therefore \ \phi_{ab} = \phi_a\centerdot\phi_b \]} \\  &= \phi_a\centerdot\phi_b \\&= (a)f\ (b)f
    \end{align*}
\item[(C)] \begin{align*}
(1)f\mathnote{\[ \text{for each}\  x \in M \ \text{we have,}\] \begin{align*} (x)\phi_{1} &= x(1) \\&=x \end{align*}\[ \therefore \ \phi_{1} \ \text{is identity of}End \ M \]} &= \phi_{1}
\end{align*}
\end{description}
\bigskip
Thus, Forward Part is proved.\newline\bigskip
\item[(Converse Part)] 
\bigskip
Assume that $\exists$ a ring homo.( say $f$) 
\[ f: R\xrightarrow[\text{Homo}]{\text{Ring}} End \ M \]
for any $a \in R$,we denote the $(a)f$ by $f_a \in End \ M$\newline
\textbf{Claim:} $M$ is a right $R$-module.\newline


\end{description}
\end{proof}
