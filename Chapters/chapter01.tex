\chapter{Introduction to Modules}
\section{\texttt{Defination of Module}}
\subsection{Left Module:}\label{sec:left-module}\index{Left Module}
 Let $ R $ be a ring with identity and $ M $ be an abelian group with addition. We say $ M $ is a left $R-$module if there exists a mapping\footnote{often called as scaler multiplication.}
\begin{equation*}
R \times M \rightarrow M
\end{equation*}	
defined by 
\begin{equation*}
(a \ , \ x) \rightarrow ax  
\end{equation*}
for each $a$ $\in$ $R$ and $x$ $\in$ $M$ satisfying following properties :
\marginnote{$\forall$ $ a\ , \ b \in R$ and $x \ , y \ \in M$  }
\begin{eqnarray}
(a+b) x &=& ax + bx\label{1} \\ a(x+y) &=& ax + ay\label{2} \\ (ab)x &=& a(bx)\label{3} \\ 1x &=& x\label{4}
\end{eqnarray}
and denoted by $_{R}M$
\subsection{Right Module:}\label{sec:right-module}\index{Right Module}
Let $ R $ be a ring with identity and $ M $ be an abelian group with addition. We say $ M $ is a right $R-$module if there exists a mapping
\begin{equation*}
M \times R \rightarrow M
\end{equation*}	
defined by 
\begin{equation*}
(x \ ,\ a) \rightarrow xa  
\end{equation*}
for each $a$ $\in$ $R$ and $x$ $\in$ $M$ satisfying following properties\marginnote{$\forall$ $ a\ , \ b \in R$ and $x \ , y \ \in M$  } :
\begin{eqnarray}
x (a+b)  &=& xa + xb\label{5} \\ (x+y) a &=& xa + ya\label{6} \\ x (ab) &=& (xa)b\label{7} \\ x1 &=& x\label{8}
\end{eqnarray}
and denoted by $M_{R}$

\subsection{Examples :}
\begin{enumerate}
\item Let $V$ be a vector space over a field $F$ then $V$ is a left as well as right $F-$Module.\newline \bigskip
\item Let $G$ be any abelian group under addition , then $G$ is a $\mathbb{Z-}$Module where $\mathbb{Z}$ is set of integers.\newline \bigskip
\item Let $R$ be ring and $M= R[x]$ \marginnote{Suppose ring $R$ is a field then $R-$Module $R[x]$ is a vector space over field $R$.} where $R[x]$ is a group of all polynomials with coefficents in $R$ then $M$ is a left as well as a right $R-$Module with scaler multiplication being usual multiplication.\newline \bigskip
\item Let $M$ be collection of all $m \times n $ matrices over ring $R$ , then $M$ is left $R-$Module where scaler multiplication being usual multiplication of a scaler to a matrix. \newline
\bigskip
In particular, if $M$ is a set of $1 \times n$ matrices over $R$ or $M = R^n$(set of $n-$tuples) then $R^n$ is a left $R-$module.
\end{enumerate}
\bigskip
\begin{remark}
	Let $R$ be a commutative ring then every left $R-$module can be transformed to right $R-$module and vice-versa.
	\end{remark}
\begin{proof} Let $M$ be left $R-$module and $R$ be a commutative ring. \newline
	so, $\exists$ a mapping
\begin{equation*}
R \times M \rightarrow M
\end{equation*}	
defined by 
\begin{equation*}
(a \ , \ x) \rightarrow ax  
\end{equation*}
for each $a$ $\in$ $R$ and $x$ $\in$ $M$ satisfying following properties :
\marginnote{$\forall$ $ a\ , \ b \in R$ and $x \ , y \ \in M$  }
\begin{eqnarray*}
(a+b) x &=& ax + bx \\ a(x+y) &=& ax + ay \\ (ab)x &=& a(bx) \\ 1x &=& x
\end{eqnarray*}
$\because \ R$ is a commutative ring.\newline
Now, Define an another mapping
 \begin{equation*}
 M \times R \rightarrow M
 \end{equation*}	
 defined by 
 \begin{equation*}
 (x \ ,\ a) \rightarrow x*a \ = \ ax  
 \end{equation*}
To check $M$ is a right $R-$Module , we need to verify properties number \eqref{5}-\eqref{8}
 \begin{enumerate}
 	\item 
 	\begin{align*}
 		x*(a+b) &= (a+b)x \\ &= ax + bx \\ &= (x*a) + (x*b)
 	\end{align*}
 	\item \begin{eqnarray*}
 		(x+y)*a &=& a(x+y) \\ &=& ax + ay \\ &=& (x*a) + (y*a)
 	      \end{eqnarray*}
       \item \begin{eqnarray*}
       x*(ab) &=& (ab)x \\ &=& (ba)x \\ &=& b(ax) \\ &=& (ax)*b
       \end{eqnarray*} 
   \item \begin{eqnarray*}
   x*1 &= 1x \\  &= x
   \end{eqnarray*}
Thus, $_{R}M$ is transformed to $M_R$.\newline
Similarly, Converse statement can be verified.
\end{enumerate}
\end{proof} \bigskip
\begin{remark}
	Let $S$ be a subring of ring $R$ then $_{S}M$ exists \marginnote{by existance means $M$ is a valid left module over mentioned ring or subring. i.e. satisfying those four properties.}  only if $_{R}M$ exists.
\end{remark} 

      
       
 	
 

