\chapter{Introduction to Modules}
\section{\texttt{Defination of Module}}
\subsection{Left Module:}\label{sec:left-module}\index{Left Module}
 Let $ R $ be a ring with identity and $ M $ be an abelian group with addition. We say $ M $ is a left $R-$module if there exists a mapping\footnote{often called as scaler multiplication.}
\begin{equation*}
R \times M \rightarrow M
\end{equation*}	
defined by 
\begin{equation*}
(a \ , \ x) \rightarrow ax  
\end{equation*}
for each $a$ $\in$ $R$ and $x$ $\in$ $M$ satisfying following properties :
\marginnote{$\forall$ $ a\ , \ b \in R$ and $x \ , y \ \in M$  }
\begin{eqnarray}
(a+b) x &=& ax + bx\label{1} \\ a(x+y) &=& ax + ay\label{2} \\ (ab)x &=& a(bx)\label{3} \\ 1x &=& x\label{4}
\end{eqnarray}
and denoted by $_{R}M$
\subsection{Right Module:}\label{sec:right-module}\index{Right Module}
Let $ R $ be a ring with identity and $ M $ be an abelian group with addition. We say $ M $ is a right $R-$module if there exists a mapping
\begin{equation*}
M \times R \rightarrow M
\end{equation*}	
defined by 
\begin{equation*}
(x \ ,\ a) \rightarrow xa  
\end{equation*}
for each $a$ $\in$ $R$ and $x$ $\in$ $M$ satisfying following properties\marginnote{$\forall$ $ a\ , \ b \in R$ and $x \ , y \ \in M$  } :
\begin{eqnarray}
x (a+b)  &=& xa + xb\label{5} \\ (x+y) a &=& xa + ya\label{6} \\ x (ab) &=& (xa)b\label{7} \\ x1 &=& x\label{8}
\end{eqnarray}
and denoted by $M_{R}$

\subsection{Examples :}
\begin{enumerate}
\item Let $V$ be a vector space over a field $F$ then $V$ is a left as well as right $F-$Module.\newline \bigskip
\item Let $G$ be any abelian group under addition , then $G$ is a $\mathbb{Z-}$Module where $\mathbb{Z}$ is set of integers.\newline \bigskip
\item Let $R$ be ring and $M= R[x]$ \marginnote{Suppose ring $R$ is a field then $R-$Module $R[x]$ is a vector space over field $R$.} where $R[x]$ is a group of all polynomials with coefficents in $R$ then $M$ is a left as well as a right $R-$Module with scaler multiplication being usual multiplication.\newline \bigskip
\item Let $M$ be collection of all $m \times n $ matrices over ring $R$ , then $M$ is left $R-$Module where scaler multiplication being usual multiplication of a scaler to a matrix. \newline
\bigskip
In particular, if $M$ is a set of $1 \times n$ matrices over $R$ or $M = R^n$(set of $n-$tuples) then $R^n$ is a left $R-$module.
\end{enumerate}
\bigskip
\begin{remark}
	Let $R$ be a commutative ring then every left $R-$module can be transformed to right $R-$module and vice-versa.
	\end{remark}
\begin{proof} Let $M$ be left $R-$module and $R$ be a commutative ring. \newline
	so, $\exists$ a mapping
\begin{equation*}
R \times M \rightarrow M
\end{equation*}	
defined by 
\begin{equation*}
(a \ , \ x) \rightarrow ax  
\end{equation*}
for each $a$ $\in$ $R$ and $x$ $\in$ $M$ satisfying following properties :
\marginnote{$\forall$ $ a\ , \ b \in R$ and $x \ , y \ \in M$  }
\begin{eqnarray*}
(a+b) x &=& ax + bx \\ a(x+y) &=& ax + ay \\ (ab)x &=& a(bx) \\ 1x &=& x
\end{eqnarray*}
$\because \ R$ is a commutative ring.\newline
Now, Define an another mapping
 \begin{equation*}
 M \times R \rightarrow M
 \end{equation*}	
 defined by 
 \begin{equation*}
 (x \ ,\ a) \rightarrow x*a \ = \ ax  
 \end{equation*}
To check $M$ is a right $R-$Module , we need to verify properties number \eqref{5}-\eqref{8}
 \begin{enumerate}
 	\item 
 	\begin{align*}
 		x*(a+b) &= (a+b)x \\ &= ax + bx \\ &= (x*a) + (x*b)
 	\end{align*}
 	\item \begin{eqnarray*}
 		(x+y)*a &=& a(x+y) \\ &=& ax + ay \\ &=& (x*a) + (y*a)
 	      \end{eqnarray*}
       \item \begin{eqnarray*}
       x*(ab) &=& (ab)x \\ &=& (ba)x \\ &=& b(ax) \\ &=& (ax)*b
       \end{eqnarray*} 
   \item \begin{eqnarray*}
   x*1 &= 1x \\  &= x
   \end{eqnarray*}
Thus, $_{R}M$ is transformed to $M_R$.\newline
Similarly, Converse statement can be verified.
\end{enumerate}
\end{proof} \bigskip
\begin{remark}
	Let $S$ be a subring of ring $R$ then $_{S}M$ exists \marginnote{by existance means $M$ is a valid left module over mentioned ring or subring. i.e. satisfying those four properties.}  only if $_{R}M$ exists.
\end{remark} \bigskip 
\begin{remark}
	Same Abelian group\marginnote{For Instance,  The field $\mathbb{R}$ is $\mathbb{R}-$module,$\mathbb{Q}-$module and $\mathbb{Z}-$module.} can have the structure of a Module for a number of different rings. 
\end{remark} \bigskip 
\begin{remark}
Let $I$ be left ideal of $R$ then quotient ring $\bigslant{R}{I}$ is a left $R$-module.\marginnote{Here scaler multiplication is \newline \begin{equation*} 
	R \times \bigslant{R}{I} \rightarrowtail \bigslant{R}{I} 
	\end{equation*}
defined as 
\begin{equation*}
(a \ , \ x+I) \rightarrowtail ax+I
\end{equation*}

$\forall \ a \in R$ and $\forall \ x+I \in \bigslant{R}{I}$} 

\end{remark} 
\begin{proof}[verification:]
	`left to reader'\newline \bigskip
	\textbf{Hint:} you need to verify those four properties: \eqref{1}-\eqref{4}
\end{proof}
\bigskip
\begin{theorem}{(Elementry Properties:)}\newline
	Let $M$ be a left $R$-module . Suppose $0_m \ \text{and} \ 0_r$ denotes additive identities of $M$ and $R$ respectively. Then,  for each $x \ \in M$ and $r \ \in R$ \newline
\begin{description}
\item (i)
	\begin{equation*}
	0_m = 0_r\ x = r\ 0_m
	\end{equation*}
\item (ii)
	\begin{equation*}
	r(-x) =  (-r)x = -rx 
	\end{equation*}
\end{description}
\end{theorem}
\begin{proof}
	\begin{description}
		\item[(i)]
As $0_m$ is the additive identity of $M$. so, $0_m = 0_m + 0_m$\newline
Consider \marginnote{$ \because \ (r\ ,\ 0_m)\rightarrowtail r\ 0_m \in M$\newline so, $r\ 0_m = r\ 0_m + 0_m$}
 $ r(0_m + 0_m) = r\ 0_m  = r\ 0_m +0_m $\newline
but, $r(0_m + 0_m) = (r\ 0_m) +(r\ 0_m)$ \marginnote{
	$ \because \ M$ is a left $R$-module.\newline (using distribuitive property)}\newline
so, we have \begin{equation*}
r\ 0_m +r\ 0_m = r\ 0_m + 0_m
\end{equation*}
as ($M,+$) is an abelian group so left and right cancellation law holds.
\begin{eqnarray*}
\cancel{r\ 0_m} +r\ 0_m &=& \cancel{r\ 0_m} + 0_m \\ r\ 0_m &=& 0_m
\end{eqnarray*}
a similiar argument can be used to prove $0_m = 0_r\ x$.
\item[(ii)]
as $M$ is a left $R$-module so $(r \ , x )\ \rightarrowtail rx \in M$ \newline
Now, Consider $(-r)x + rx$
\begin{eqnarray*}
\text{using distribuitive law} \\ (-r)x + rx &=& (-r + r)x \\  &=& 0_r \ x \\ &=& 0_m
\end{eqnarray*}
i.e. $(-r)x$ is additive inverse of  $(rx)$ but additive inverse of $(rx)$ is $-rx$ and it is unique for an abelian group($M$ here)
\begin{equation*}
\therefore \ (-r)x = -rx
\end{equation*}
a similar argument can be used to prove that $r(-x)  = -r x$.
\end{description}
\end{proof}
\bigskip
\begin{definition}[Ring Homomorphism]
	Let $R$ and $S$ be two rings with identities $1_r \ , 1_s$ respectively then a map(say $f$)
	\begin{equation*}
	f : R \rightarrow S
	\end{equation*}
	is said to be a ring homomorphism (often said ring homo) or ring linear map if for every $a \ ,\ b \in R$ following properties holds
	\begin{description}
		\item[(i)]\begin{equation*}
		(a+b)f = (a)f +(b)f
		\end{equation*}
		\item[(ii)]\begin{equation*}
		(ab)f = (a)f \ (b)f
		\end{equation*}
		\item[(iii)]\begin{equation*}
		(1_r)f = 1_s
		\end{equation*}
	\end{description}
\end{definition}
