\documentclass{tufte-book}
\hypersetup{colorlinks}

\title{Module Theory\thanks{Thanks to Prof. Anuj Bishnoi  for his inspiration.}}
\author[Kapil Chaudhary]{Kapil Chaudhary}
\publisher{University of Delhi}

\usepackage{amsmath}
\usepackage{amsthm}
\usepackage{amsfonts}
\usepackage{amssymb}
\usepackage{lipsum}
\usepackage{hyperref}
%%
% For nicely typeset tabular material
\usepackage{booktabs}
\setcounter{section}{1}
\setcounter{chapter}{1}
\renewcommand{\theequation}{\thesection.\arabic{equation}}
\renewcommand{\thesection}{\arabic{section}}
\renewcommand{\thesubsection}{(\alph{subsection})}
%%
\renewcommand\qedsymbol{$\blacksquare$}
\theoremstyle{definition}
\newtheorem{definition}{Definition}[section]

\theoremstyle{theorem}
\newtheorem{thm}{Theorem}[section]
\newtheorem{lemma}[thm]{Lemma}
\newtheorem{prop}[thm]{Proposition}
\newtheorem{cor}[thm]{Corollary}
\newtheorem{defn}[thm]{Definition}
\newtheorem{examp}[thm]{Example}
\newtheorem{conj}[thm]{Conjecture}
\newtheorem{remark}[thm]{Remark:}
\numberwithin{equation}{chapter}
\DeclareRobustCommand{\eqref}[1]{\eqrefaux#1\eqrefaux}
\def\eqrefaux eq#1\eqrefaux{\textup{(#1)}}
% For graphics / images
\usepackage{graphicx}
\setkeys{Gin}{width=\linewidth,totalheight=\textheight,keepaspectratio}
\graphicspath{{graphics/}}

% The fancyvrb package lets us customize the formatting of verbatim
% environments.  We use a slightly smaller font.
\usepackage{fancyvrb}
\fvset{fontsize=\normalsize}

%%
% Prints argument within hanging parentheses (i.e., parentheses that take
% up no horizontal space).  Useful in tabular environments.
\newcommand{\hangp}[1]{\makebox[0pt][r]{(}#1\makebox[0pt][l]{)}}

%%
% Prints an asterisk that takes up no horizontal space.
% Useful in tabular environments.
\newcommand{\hangstar}{\makebox[0pt][l]{*}}

%%
% Prints a trailing space in a smart way.
\usepackage{xspace}
\usepackage{cancel}
%%
% Some shortcuts for Tufte's book titles.  The lowercase commands will
% produce the initials of the book title in italics.  The all-caps commands
% will print out the full title of the book in italics.
\newcommand{\vdqi}{\textit{VDQI}\xspace}
\newcommand{\ei}{\textit{EI}\xspace}
\newcommand{\ve}{\textit{VE}\xspace}
\newcommand{\be}{\textit{BE}\xspace}
\newcommand{\VDQI}{\textit{The Visual Display of Quantitative Information}\xspace}
\newcommand{\EI}{\textit{Envisioning Information}\xspace}
\newcommand{\VE}{\textit{Visual Explanations}\xspace}
\newcommand{\BE}{\textit{Beautiful Evidence}\xspace}

\newcommand{\TL}{Tufte-\LaTeX\xspace}
\newcommand{\bigslant}[2]{{\raisebox{.1em}{$#1$}\left/\raisebox{-.2em}{$#2$}\right.}}
% Prints the month name (e.g., January) and the year (e.g., 2008)
\newcommand{\monthyear}{%
	\ifcase\month\or January\or February\or March\or April\or May\or June\or
	July\or August\or September\or October\or November\or
	December\fi\space\number\year
}


% Prints an epigraph and speaker in sans serif, all-caps type.
\newcommand{\openepigraph}[2]{%
	%\sffamily\fontsize{14}{16}\selectfont
	\begin{fullwidth}
		\sffamily\large
		\begin{doublespace}
			\noindent\allcaps{#1}\\% epigraph
			\noindent\allcaps{#2}% author
		\end{doublespace}
	\end{fullwidth}
}

% Inserts a blank page
\newcommand{\blankpage}{\newpage\hbox{}\thispagestyle{empty}\newpage}

\usepackage{units}
\def\mathnote#1{%
  \tag*{\rlap{\hspace\marginparsep\smash{\parbox[t]{\marginparwidth}{%
  \footnotesize#1}}}}
}% Typesets the font size, leading, and measure in the form of 10/12x26 pc.
\newcommand{\measure}[3]{#1/#2$\times$\unit[#3]{pc}}

% Macros for typesetting the documentation
\newcommand{\hlred}[1]{\textcolor{Maroon}{#1}}% prints in red
\newcommand{\hangleft}[1]{\makebox[0pt][r]{#1}}
\newcommand{\hairsp}{\hspace{1pt}}% hair space
\newcommand{\hquad}{\hskip0.5em\relax}% half quad space
\newcommand{\TODO}{\textcolor{red}{\bf TODO!}\xspace}
\newcommand{\na}{\quad--}% used in tables for N/A cells
\providecommand{\XeLaTeX}{X\lower.5ex\hbox{\kern-0.15em\reflectbox{E}}\kern-0.1em\LaTeX}
\newcommand{\tXeLaTeX}{\XeLaTeX\index{XeLaTeX@\protect\XeLaTeX}}
% \index{\texttt{\textbackslash xyz}@\hangleft{\texttt{\textbackslash}}\texttt{xyz}}
\newcommand{\tuftebs}{\symbol{'134}}% a backslash in tt type in OT1/T1
\newcommand{\doccmdnoindex}[2][]{\texttt{\tuftebs#2}}% command name -- adds backslash automatically (and doesn't add cmd to the index)
\newcommand{\doccmddef}[2][]{%
	\hlred{\texttt{\tuftebs#2}}\label{cmd:#2}%
	\ifthenelse{\isempty{#1}}%
	{% add the command to the index
		\index{#2 command@\protect\hangleft{\texttt{\tuftebs}}\texttt{#2}}% command name
	}%
	{% add the command and package to the index
		\index{#2 command@\protect\hangleft{\texttt{\tuftebs}}\texttt{#2} (\texttt{#1} package)}% command name
		\index{#1 package@\texttt{#1} package}\index{packages!#1@\texttt{#1}}% package name
	}%
}% command name -- adds backslash automatically
\newcommand{\doccmd}[2][]{%
	\texttt{\tuftebs#2}%
	\ifthenelse{\isempty{#1}}%
	{% add the command to the index
		\index{#2 command@\protect\hangleft{\texttt{\tuftebs}}\texttt{#2}}% command name
	}%
	{% add the command and package to the index
		\index{#2 command@\protect\hangleft{\texttt{\tuftebs}}\texttt{#2} (\texttt{#1} package)}% command name
		\index{#1 package@\texttt{#1} package}\index{packages!#1@\texttt{#1}}% package name
	}%
}% command name -- adds backslash automatically
\newcommand{\docopt}[1]{\ensuremath{\langle}\textrm{\textit{#1}}\ensuremath{\rangle}}% optional command argument
\newcommand{\docarg}[1]{\textrm{\textit{#1}}}% (required) command argument
\newenvironment{docspec}{\begin{quotation}\ttfamily\parskip0pt\parindent0pt\ignorespaces}{\end{quotation}}% command specification environment
\newcommand{\docenv}[1]{\texttt{#1}\index{#1 environment@\texttt{#1} environment}\index{environments!#1@\texttt{#1}}}% environment name
\newcommand{\docenvdef}[1]{\hlred{\texttt{#1}}\label{env:#1}\index{#1 environment@\texttt{#1} environment}\index{environments!#1@\texttt{#1}}}% environment name
\newcommand{\docpkg}[1]{\texttt{#1}\index{#1 package@\texttt{#1} package}\index{packages!#1@\texttt{#1}}}% package name
\newcommand{\doccls}[1]{\texttt{#1}}% document class name
\newcommand{\docclsopt}[1]{\texttt{#1}\index{#1 class option@\texttt{#1} class option}\index{class options!#1@\texttt{#1}}}% document class option name
\newcommand{\docclsoptdef}[1]{\hlred{\texttt{#1}}\label{clsopt:#1}\index{#1 class option@\texttt{#1} class option}\index{class options!#1@\texttt{#1}}}% document class option name defined
\newcommand{\docmsg}[2]{\bigskip\begin{fullwidth}\noindent\ttfamily#1\end{fullwidth}\medskip\par\noindent#2}
\newcommand{\docfilehook}[2]{\texttt{#1}\index{file hooks!#2}\index{#1@\texttt{#1}}}
\newcommand{\doccounter}[1]{\texttt{#1}\index{#1 counter@\texttt{#1} counter}}

% Generates the index
\usepackage{makeidx}
\makeindex


\begin{document}
\frontmatter

\maketitle
\newpage
\begin{fullwidth}
	~\vfill
	\thispagestyle{empty}
	\setlength{\parindent}{0pt}
	\setlength{\parskip}{\baselineskip}
	Copyright \copyright\ \the\year\ , All rights are reserved. \newline \thanklessauthor
		
	\par\smallcaps{\thanklesspublisher}
	
	\par\texttt{\url{https://contact.sirkapil.me/}}
	\par Licensed under the Apache License, Version 2.0 (the ``License''); you may not
	use this file except in compliance with the License. You may obtain a copy
	of the License at \url{http://www.apache.org/licenses/LICENSE-2.0}. Unless
	required by applicable law or agreed to in writing, software distributed
	under the License is distributed on an \smallcaps{``AS IS'' BASIS, WITHOUT
		WARRANTIES OR CONDITIONS OF ANY KIND}, either express or implied. See the
	License for the specific language governing permissions and limitations
	under the License.\index{license}
	\par\textit{First Print, \monthyear}
\end{fullwidth}

\tableofcontents
\cleardoublepage
\begin{fullwidth}
~\vfill
\begin{doublespace}
	\noindent\fontsize{18}{22}\selectfont\itshape
	\nohyphenation
   There are few persons without whom this was impossible.\newline
   They deserve credit for it so i would love to thank them.\newline
\vfill
	Special Thanks to :\newline
        \begin{center}
        -  Dr. Anuj Bishnoi (Subject \ Teacher)\newline
		-	Edward Tufte (\LaTeX \  Tufte  Templete)
		\end{center}
\end{doublespace}
\vfill
\vfill
\end{fullwidth}
\newpage
\begin{fullwidth}
~\vfill
\vfill
\begin{doublespace}
	\noindent\fontsize{18}{22}\selectfont\itshape
	\nohyphenation
	\centering
	Dedicated to my family and my best friend\newline\texttt{Neeraj K. Gaud}
\end{doublespace}
\vfill
\vfill
\end{fullwidth}
\newpage

\chapter*{Introduction}
\section{Warning :}
This is my first document created using latex so it may be possible that there are typo errors and other errors. if you notice any such error then you can report it here.\newline
\ \url{https://github.com/sirkapil/module-theory/issues/new}\footnote{(may require a github account)}
\section{About :}
This sample book discusses the course "Module Theory" being taught to Post-Graduate (M.Sc. Mathematics) students in Department of Mathematics under University of Delhi, Delhi.\newline\bigskip
All my \ \LaTeX \ documents are free, open-source and can be found pinned here :\newline
\url{https://github.com/sirkapil} 

\section{Contribution :}
any contribution will be welcomed.


%%
% Start the main matter (normal chapters)
\mainmatter
\chapter{Introduction to Modules}
\section{\textbf{Definition of Module}}
\begin{defn}  [\textbf{Left Module}]

 Let $ R $ be a ring with identity and $ M $ be an abelian group with addition. We say $ M $ is a left $R-$module if there exists a mapping\footnote{often called as scaler multiplication.}
\begin{equation*}
R \times M \rightarrow M
\end{equation*}	
defined by
\begin{equation*}
(a \ , \ x) \rightarrow ax
\end{equation*}\marginnote[-2em]{$\forall \ a \in \ R$ and $x$ $\in \ M$}
satisfying following properties :
\begin{align}
(a+b) x =& ax + bx \\ a(x+y) =& ax + ay \\ (ab)x =& a(bx) \\ 1x =& x
\end{align}\marginnote[-7em]{\large{$\forall$ $ a\ , \ b \in R$ \newline $x \ , y \ \in M$}}
and denoted by $_{R}M$
\end{defn}
\bigskip
\begin{defn}[\textbf{Right Module}]
 Let $ R $ be a ring with identity and $ M $ be an abelian group with addition. We say $ M $ is a right $R-$module if there exists a mapping
\begin{equation*}
M \times R \rightarrow M
\end{equation*}	
defined by
\begin{equation*}
(x \ ,\ a) \rightarrow xa
\end{equation*}
\marginnote[-1em]{$ \forall \ a$ $\in$ $R$ and $x$ $\in$ $M$} satisfying following properties\marginnote[4em]{\large{$\forall$ $ a\ , \ b \in R$ \newline $x \ , y \ \in M$}} :
\begin{align}
x (a+b)  &= xa + xb \\ (x+y) a &= xa + ya \\ x (ab) &= (xa)b \\ x1 &= x
\end{align}
and denoted by $M_{R}$.         \end{defn}

\subsection{Examples :}
\begin{enumerate}
\item Let $V$ be a vector space over a field $F$ then $V$ is a left as well as right $F-$Module.\newline \bigskip
\item Let $G$ be any abelian group under addition , then $G$ is a $\mathbb{Z-}$Module where $\mathbb{Z}$ is set of integers.\newline \bigskip
\item Let $R$ be ring and $M= R[x]$ \marginnote{Suppose ring $R$ is a field then $R-$Module $R[x]$ is a vector space over field $R$.} where $R[x]$ is a group of all polynomials with coefficents in $R$ then $M$ is a left as well as a right $R-$Module with scaler multiplication being usual multiplication.\newline \bigskip
\item Let $M$ be collection of all $m \times n $ matrices over ring $R$ , then $M$ is left $R-$Module where scaler multiplication being usual multiplication of a scaler to a matrix. \newline
\bigskip
In particular, if $M$ is a set of $1 \times n$ matrices over $R$ or $M = R^n$(set of $n-$tuples) then $R^n$ is a left $R-$module.
\end{enumerate}
\bigskip
\begin{remark}
	Let $R$ be a commutative ring then every left $R-$module can be transformed to right $R-$module and vice-versa.
	\end{remark}
\begin{proof} Let $M$ be left $R-$module and $R$ be a commutative ring. \newline
	so, $\exists$ a mapping
\begin{equation*}
R \times M \rightarrow M
\end{equation*}	
defined by
\begin{equation*}
(a \ , \ x) \rightarrow ax
\end{equation*}
for each $a$ $\in$ $R$ and $x$ $\in$ $M$ satisfying following properties :
\marginnote{$\forall$ $ a\ , \ b \in R$ and $x \ , y \ \in M$  }
\begin{eqnarray*}
(a+b) x &=& ax + bx \\ a(x+y) &=& ax + ay \\ (ab)x &=& a(bx) \\ 1x &=& x
\end{eqnarray*}
$\because \ R$ is a commutative ring.\newline
Now, Define an another mapping
 \begin{equation*}
 M \times R \rightarrow M
 \end{equation*}	
 defined by
 \begin{equation*}
 (x \ ,\ a) \rightarrow x*a \ = \ ax
 \end{equation*}
To check $M$ is a right $R-$Module , we need to verify properties number \eqref{eq1.5}-\eqref{eq1.8}
 \begin{description}
 	\item[(i) Distribuitive Law]
 	\begin{align*}
 		x*(a+b) &= (a+b)x \\ &= ax + bx \\ &= (x*a) + (x*b)
 	\end{align*}
 	\item[(ii) Distributive Law] \begin{align*}
 		(x+y)*a &= a(x+y) \\ &= ax + ay \\ &= (x*a) + (y*a)
 	      \end{align*}
       \item[(iii)]  \begin{align*}
       x*(ab) &= (ab)x \\ &= (ba)x \\ &= b(ax) \\ &= (ax)*b
       \end{align*}
   \item[(iv)] \begin{align*}
   x*1 &= 1x \\  &= x
   \end{align*}
Thus, $_{R}M$ is transformed to $M_R$.\newline
Similarly, Converse statement can be verified.
\end{description}
\end{proof} \bigskip
\begin{remark}
	Let $S$ be a subring of ring $R$ then $_{S}M$ exists \marginnote{by existance means $M$ is a valid left module over mentioned ring or subring. i.e. satisfying those four properties.}  only if $_{R}M$ exists.
\end{remark} \bigskip
\begin{remark}
	Same Abelian group\marginnote{For Instance,  The field $\mathbb{R}$ is $\mathbb{R}-$module,$\mathbb{Q}-$module and $\mathbb{Z}-$module.} can have the structure of a Module for a number of different rings.
\end{remark} \bigskip
\begin{remark}
Let $I$ be left ideal of $R$ then quotient ring $\bigslant{R}{I}$ is a left $R$-module.\marginnote{Here scaler multiplication is \newline \begin{equation*}
	R \times \bigslant{R}{I} \rightarrow \bigslant{R}{I}
	\end{equation*}
defined as
\begin{equation*}
(a \ , \ x+I) \rightarrow ax+I
\end{equation*}

$\forall \ a \in R$ and $\forall \ x+I \in \bigslant{R}{I}$}

\end{remark}
\begin{proof}[verification:]
	Left to reader\newline \bigskip
	\textbf{Hint:} you need to verify those four properties: \eqref{eq1.1}-\eqref{eq1.4}
\end{proof}
\bigskip

\begin{thm}{\textbf{(Elementry Properties:)}}\newline
	Let $M$ be a left $R$-module . Suppose $0_m \ \text{and} \ 0_r$ denotes additive identities of $M$ and $R$ respectively. Then,  for each $x \ \in M$ and $r \ \in R$ \newline
\begin{description}
\item (i)
	\begin{equation*}
	0_m = 0_r\ x = r\ 0_m
	\end{equation*}
\item (ii)
	\begin{equation*}
	r(-x) =  (-r)x = -rx
	\end{equation*}
\end{description}
\end{thm}
\begin{proof}
	\begin{description}
		\item[(i)]
As $0_m$ is the additive identity of $M$. so, $0_m = 0_m + 0_m$\newline
Consider \marginnote{$ \because \ (r\ ,\ 0_m)\rightarrow r\ 0_m \in M$\newline so, $r\ 0_m = r\ 0_m + 0_m$}
 $ r(0_m + 0_m) = r\ 0_m  = r\ 0_m +0_m $\newline
but, $r(0_m + 0_m) = (r\ 0_m) +(r\ 0_m)$ \marginnote{
	$ \because \ M$ is a left $R$-module.\newline (using distribuitive property)}\newline
so, we have \begin{equation*}
r\ 0_m +r\ 0_m = r\ 0_m + 0_m
\end{equation*}
as ($M,+$) is an abelian group so left and right cancellation law holds.
\begin{eqnarray*}
\cancel{r\ 0_m} +r\ 0_m &=& \cancel{r\ 0_m} + 0_m \\ r\ 0_m &=& 0_m
\end{eqnarray*}
a similiar argument can be used to prove $0_m = 0_r\ x$.
\item[(ii)]
as $M$ is a left $R$-module so $(r \ , x )\ \rightarrow rx \in M$ \newline
Now, Consider $(-r)x + rx$
\begin{eqnarray*}
\text{using distribuitive law} \\ (-r)x + rx &=& (-r + r)x \\  &=& 0_r \ x \\ &=& 0_m
\end{eqnarray*}
i.e. $(-r)x$ is additive inverse of  $(rx)$ but additive inverse of $(rx)$ is $-rx$ and it is unique for an abelian group($M$ here)
\begin{equation*}
\therefore \ (-r)x = -rx
\end{equation*}
a similar argument can be used to prove that $r(-x)  = -r x$.
\end{description}
\end{proof}
\bigskip
\begin{defn}[ \textbf{Ring Homomorphism}]
	Let\marginnote{often called as ring homo} $R$ and $S$ be two rings with identities $1_r \ , 1_s$ respectively then a map(say $f$)
	\begin{equation*}
	f : R \rightarrow S
	\end{equation*}
	is said to be a ring homomorphism or ring linear map if for every $a \ ,\ b \in R$ following properties holds\marginnote[-2.55em]{if $R$ = $S$ then we call ring homo as ring endomorphism. For instance , let $f$ be ring homo from $R$ to $R$ . we say $f$ is endomorphism of $R$ and denoted by $End \ R$}
	\begin{description}
		\item[(i) \centering{Preserves Addition}]\begin{equation*}
		(a+b)f = (a)f +(b)f
		\end{equation*}
		\item[(ii) \centering{Preservers Multiplication} ]\begin{equation*}
		(ab)f = (a)f . (b)f
		\end{equation*}
		\item[(iii) \centering{ Maps identity to identity}]\begin{equation*}
		(1_r)f = 1_s
		\end{equation*}
	\end{description}
\end{defn}
\begin{remark}
	Such a mapping need not to be bijective. if it is bijective then we say it is a ring isomorphism or rings are isomorphic.
\end{remark}
\bigskip
\begin{thm}
	Let\marginnote{\[ M \ \text{is a right R-module} \] \[\centering\Updownarrow \] \[\newline\exists \ f: R \ \xrightarrow[\text{Homo}]{\text{Ring}} End \ M \] } $R$ be a ring and $M$ be any abelian group with addition. then $M$ is a right $R$-module if and only if there exists a map which is ring homomorphism from $R$ to $End \ M$
\end{thm}
\begin{proof}
\begin{description}
	\item[(Forward Part)\newline]
	Let us suppose that $ M $ is a right $ R $-module.\newline
	\textbf{Claim:} there exists a map which is ring homomorphism from $R$ to $End \ M$\newline
   $ \because \ M  $ is a left $ R $-module , so there exist a map\newline
   \[ f: M \times R \rightarrow M  \]
   defined by
   \[ (x \ , a) \rightarrow ax \]
   satisfying following properties:
    \begin{align*} (x+y)a &= (x)a+(y)a \\
     x(a+b)\mathnote{ $$\forall \ x , y \ \in M \ \&  \ a ,b \in R$$ } &= xa + xb \\
     x(ab) &= (xa)b \\
     x 1 &= x
     \end{align*}
   for each $ a \in R $ , define a map(say $\phi_a $)
   \[ \phi_a: M\rightarrow\ M \]
   such that for each $x\in  M$
   \[ (x )\phi_a = xa \in \ M \]
   Now, we'll show that $ \phi_a \in \ End \ M $\newline
   Let $ x , y \in M $ \newline Consider  $  (x+y)\phi_a$
   \begin{align*}
   	&= (x+y)a  \\ &= xa + ya  \\ &= (x)\phi_a + (y)\phi_a
   \end{align*}\marginnote[-5.5em]{using defination of $\phi_a$} \marginnote[-4em]{using \eqref{eq1.9}}
   so, $\phi_a$ preserves addition and is a group homo from $M$ to $M$.
   \newline i.e. $\phi_a \in End \ M$ \newline
   Now, we can define a map (say $f$)
   \begin{align*}
   &f : R \rightarrow End \ M \\ \text{defined as} \\&(a)f \rightarrow \phi_a \mathnote{$\forall a \in R$ and $\phi_a \in End \ M$}
   \end{align*}
Now, We'll show that $ f $ is a ring homomorphism.\newline
\begin{description}
	\item[(A)]
	  \begin{align*}
	(a+b)f &= \phi_{a+b}
	\mathnote{\[ \text{for each}\  x \in M \ \text{we have,}\] \begin{align*}
		(x)\phi_{a+b} &= x(a+b) \ =xa+xb \\ &=(x)\phi_a +(y)\phi_b \\ \therefore \ \phi_{a+b} &= \phi_a +\phi_b
		\end{align*}}
		  \\  &= \phi_a +\phi_b \\&= (a)f +(b)f
	\end{align*}
    \item[(B)] \begin{align*}
    (ab)f &= \phi_{ab}\mathnote{\[ \text{for each}\  x \in M \ \text{we have,}\] \begin{align*} (x)\phi_{ab} &= x(ab)   = (xa)b \\ &= (xa)\phi_b = (x)\phi_a \circ\phi_b \\ \therefore \ \phi_{ab} &= \phi_a\circ\phi_b \end{align*} } \\  &= \phi_a\circ\phi_b \\&= (a)f\ (b)f
    \end{align*}
\item[(C)] \begin{align*}
(1)f\mathnote{\[ \text{for each}\  x \in M \ \text{we have,}\] \begin{align*} (x)\phi_{1} &= x(1) \\&=x \end{align*}\[ \therefore \ \phi_{1} \ \text{is identity of}End \ M \]} &= \phi_{1}
\end{align*}

   \end{description}
\bigskip
Thus, Forward Part is proved.\newline\bigskip
\item[(Converse Part)]
\bigskip
Assume that $\exists$ a ring homo.( say $f$)
\[ f: R\xrightarrow[\text{Homo}]{\text{Ring}} End \ M \]
for any $a \in R$,we denote the $(a)f$ by $f_a \in End \ M$\newline
\textbf{Claim:} $M$ is a right $R$-module.\newline
so let's define a map
\[ R\times M \longrightarrow M \] defined by
\[ (a , x ) \rightarrow x*a = (x)f_a \]
to prove $M$ is a right $R$-module , we need to verify four properties $\eqref{eq1.5}$- $\eqref{eq1.9}$ of right $R$-module.
\begin{description}
	\item[(i)]
\end{description}
\begin{align*}
(x+y)*a &= (x+y)f_a \\& = (x)f_a +(y)f_a \mathnote{\[ \because\ f_a \ \in End \ M \]}\\& = x*a + y*a
\end{align*}
   \item{(ii)}
   \begin{align*}
   x*(a+b) &= (x)f_{a+b} \\ &= (x)(f_a + f_b) \mathnote{\[ \because\ f_a \ , \ f_b \in End \ M \]} \\&=(x)f_a + (x)f_b
 \\ &= (x*a) + (x*b)   \end{align*}
 \item{(iii)}
 \begin{align*}
 x*(ab) &= (x)f_{ab} \\ & = (x)f_a\circ f_b \\&=(x f_a)f_b \\&= (x*a)*b
 \end{align*}
 \item{(iv)}
 \[ x*1 = (x)f_1  = x\]\marginnote{\[ \because\ f_1 \text{is identity in} End \ M \]}
Thus, $ M $ is a right $R$-module.
\end{description}
\end{proof}
\begin{definition}[\textbf{Anti-Ring Homomorphism}]
	Let $R$ and $S$ be two rings with identities $1_r$ and $1_s$ respectively. Define a map $f$
	\[ f : R \rightarrow S \] satisfying following properties, for each $a , b \in R$ \newline
	\begin{description}
		\item[(i)] \[ (a+b)f = (a)f +(b)f \]
		\item[(ii)] \[ (ab)f  = (b)f\ (a)f \]
		\item[(iii)] \[ (1_r)f = 1_s \]
  	
	\end{description}
	 Then, $f$	is called anti-ring homomorphism.
\end{definition}
\begin{thm}
	Let\marginnote{\[ M \ \text{is a left R-module} \] \[\centering\Updownarrow \] \[\newline\exists \ f: R \ \xrightarrow[\text{Homo}]{\text{Anti-Ring}} End \ M \] } $R$ be a ring and $M$ be any abelian group with addition. then $M$ is a left $R$-module if and only if there exists a map which is anti-ring homomorphism from $R$ to $End \ M$.
\end{thm}
\begin{proof}
	Left to reader.
\end{proof}
\bigskip
\begin{definition}[\textbf{SubModule}]
	Let $M$ be a left (right) $R$-module then a subset $N$ of $M$ is called a submodule of $M$ if \underline{$N$ is a left (right) $R$-module} \underline{under the operation induced from $M$.}\newline \bigskip
	In other words, A subset $N$ of $M$ is called submodule of $M$ if
	\begin{description}
		\item[(i)] $N$ is subgroup of $M$.
		\item[(ii)] $N$ is closed under induced scaler multiplication from $M$.
	\end{description}
\end{definition}
\bigskip
\begin{thm}[\textbf{Criterion for Checking Modules}]

	Let $M$ be a left (right) $R$-module and $N$ be a subset of $M$ then $N$ is a submodule of $M$ if and only if
	\begin{description}
		\item[(i)]
		\marginnote[2.5em]{\[ \forall \ x , y  \in N \] } \[ x-y \in N \]
		\item[(ii)]
		\marginnote[2.5em]{\[ \forall \ a \in R \ \&\ x \in N \]} \[ a x \in N \]
	\end{description}
\end{thm}
\begin{proof}
	Left to reader.
\end{proof}
\bigskip
\subsection{Examples:}
\begin{enumerate}
	\item As every Vector Space $V$ over a Field $F$ is a $F$-module. So, submodules of $V$ are subspaces of $V$.
	\item As every abelian group $G$ is a $\mathbb{Z}$-module. So, all subgroups of $G$ are submodules.
	\item Let $R$ be a ring then $R$ is a left as well as right $R$-module then left (right) ideals of $R$ are left (right) submodules of $R$.
	\item $\left\{0\right\}$ and $M$ are trivial submodules of any left (right) $R$-module $M$.
\end{enumerate}
\bigskip
\begin{remark}\qquad
	\begin{enumerate}
		\item Union of two submodules need not to be a submodule.\marginnote{Think an example !}
		
		\item Intersection of any number of submodules is again a submodule.\marginnote{\textbf{Hint:} Verify using criterion for checking modules.}
	\end{enumerate}
\end{remark}
\bigskip	

\begin{remark}\textbf{(Smallest Submodule containing a set)}\newline
	  		Let $M$ be any left (right) $R$-module and $S$ be any subset of $M$. Suppose $\mathcal{F}$ be the family of all submodules of $M$ containing $S$. \[  \text{Let} \; P=\bigcap_{N \in \mathcal{F}}N  \] then $P$ is a submodule of $M$ containing $S$ as being intersection of an indexed family of submodules containing $S$.\newline \bigskip Moreover, $P$ is the smallest submodule of $M$ containing $S$. i.e. for any arbitrary submodule $K \in \mathcal{F}$ , we have $ P \subseteq K$. Such submodule $P$ of $M$ is said to be generated by set $S$ and is denoted by \[ P = \left\langle S \right\rangle  = (S) \]
\end{remark}
\bigskip
\begin{remark}\qquad
	\begin{description}
	\item Let $S$ be any subset of left $R$-module $M$ and $\left\langle S \right\rangle $ is the smallest submodule of $M$ containing $S$.\newline
	\begin{enumerate}
		\item if $S$ is non-empty and finite , $S = \left\lbrace x_1,x_2,x_3, \cdots , x_n \right\rbrace $ \[ \left\langle S \right\rangle = \left\langle \left\lbrace x_1,x_2,x_3, \cdots , x_n \right\rbrace \right\rangle  = \left\langle x_1,x_2,x_3, \cdots , x_n \right\rangle \] is said to be a finitely generated by $S$ and is smallest submodule of $M$ containing $S$.
		\item if $S = \phi$ \quad i.e. $S$ is an empty set \[ \left\langle S \right\rangle = \left\langle \phi \right\rangle = \left\lbrace 0 \right\rbrace \]
		\item if $S = \left\lbrace a\right\rbrace $ \quad i.e. $S $ is singleton then $\left\langle S \right\rangle = \left\langle a\right\rangle $ is said to be a \underline{cyclic submodule}.
	\end{enumerate}
\end{description}
\end{remark}
\bigskip
\begin{definition}[\textbf{Cyclic module}]   \cite{Cohn2005}
	A module $M$ is said to be a cyclic module if it can be   generated by a single element.
\end{definition}
\textit{For Example:}
	A ring $R$ over itself is a module and can be generated by identity element $ \left\lbrace 1\right\rbrace $ so is a cyclic module.
\bigskip
\begin{thm}
	Let $M$ be left $R$ module and $S$ being any subset of $M$.
	\begin{equation*} \left\langle S\right\rangle = \begin{cases}
	\left\lbrace 0\right\rbrace & \text{if $S = \phi$} \\ \left\lbrace \sum\limits_{i \in J_n} a_i x_i \ \middle\vert \  a_i \in R \ , x_i \in S\right\rbrace  & \text{otherwise}
	\end{cases}
	\end{equation*}
\end{thm}
\begin{proof}
\begin{description}
	\item[Case-I] Let us suppose that $S = \phi $ ,as $ \left\langle S\right\rangle $ is the intersection of all the submodules of $ M $ containing $S$\marginnote[-2em]{i.e. Every submodule of $M$ will contain $S$}. \newline \bigskip In particular, $ \left\lbrace 0\right\rbrace  $ also contains $S$ i.e. \[ \left\lbrace 0\right\rbrace  \in\mathnote{$\because \ \mathcal{F} \ \text{is a collection of all submodules of}\ M \newline \text{containing}\ S$ } \mathcal{F}  \]
	so, \begin{align*} \left\langle S\right\rangle &= \bigcap_{N \in \mathcal{F}}N  \\& = \left\lbrace 0\right\rbrace   \end{align*}
    \item[Case-II] Suppose $S$ is non-empty and let    \[ P = \left\lbrace \sum\limits_{i \in J_n} a_i x_i \ \middle\vert \  a_i \in R \ , x_i \in S\right\rbrace\]
         First , we'll show that $S\subseteq P$ \newline
         Let $x \in\ S$ then it can be expressed in following form:
         \[x = 1.x = \sum\limits_{i \in J_1} a_i x_i \] \marginnote[-2.5em]{with $a_1 = 1$ and $x_1 = x $}
         \[\therefore x \in  P \Rightarrow S \subseteq P \]  \marginnote[-2em]{$\because x$ was chosen arbitirary.}
         Now ,we'll show that $P$ is a submodule of $M$ using submodule criterion.           \newline
         Let $u\ ,\ v \in P$ . so, we need to show $u + \alpha v \in P$ \marginnote{for any $\alpha\in R$}
         \[u = \sum\limits_{i\in J_n} a_i x_i\] \marginnote[-2em]{$\forall x_i \in S \ \&\ a_i \in R$}
          \[v = \sum\limits_{j\in J_m} b_j y_j\]  \marginnote[-2em]{$\forall y_j \in S \ \&\ b_j \in R$}
          define, for any $\alpha \in R$
          \begin{align*} z_k = x_k \qquad ,\qquad c_k = a_k     \mathnote{$k \in J_n$}  \\
           z_{k+j} = y_j \qquad , \qquad c_{k+j} = \alpha b_j    \mathnote{$j \in J_m$}  \end{align*}
          Thus, we have
           \begin{align*}
            u+ \alpha v &= \sum\limits_{i\in J_n}a_i x_i + \alpha \sum\limits_{j\in J_m}b_j y_j  \\
             & =\sum\limits_{i\in J_n}a_i x_i + \sum\limits_{j\in J_m}\alpha b_j y_j  \\
             &= \sum\limits_{k\in J_n}c_k z_k + \sum\limits_{k = n+1}^{n+m} c_k z_k  \\
             &=  \sum\limits_{k \in J_{n+m}} c_k z_k
          \end{align*}
          so, $P$ is a submmodule of $M$ containing $S$.
          \newline Now, we'll show that $P$ is \underline{smallest} submmodule of $M$ containing $S$.
          \newline \bigskip
          Let $K$ be any arbitirary submodule of $M$ containing $S$
           \[\text{i.e.} \ K \ \in \mathcal{F}\]
          $\because  \ \ \ K $ is a submodule and $S \subseteq K$ \newline
          $\therefore \  K$  is closed under scaler multiplication and addition.
          \[\text{i.e}\qquad \sum\limits_{i \in J_n} a_i x_i  \in K \]  \marginnote[-2.5em]{$\forall \ a_i \in R \ \& \ x_i \in S $}
          so , \[P =\left\langle S\right\rangle \subseteq K \]
               Hence,  $P$ is smallest submmodule of $M$ containing $S$.
           \end{description}	
	
	
\end{proof}
\bigskip
\begin{defn}[\textbf{Generating Set / Set of Generators}]
A set of generators for a left (right) $R$-module $M$ is a subet $S$ of $M$ such that
\[M = \left\langle S\right\rangle \]
if no proper submodule of $M$ contains $S$ then $S$ generates $M$ (verify ?)
\end{defn}
\bigskip
\subsection{Examples}
\begin{description}
  \item[1.] A   ring $R$ considered as left (right) $R$-module is generated by identity element $\lbrace1\rbrace$

  \item[2.] $\mathbb{Z}\times \mathbb{Z}$  over $ \mathbb{Z}  $ can be generated by
  \[S =\left\lbrace(0,1),(1,0)\right\rbrace\]
  \item[3.] All finite dimensional vector space can be generated by it's basis (finite), so is finitely generated submodule.
  \item[4.] Let $R$ be a ring, $I$ be left(right) ideal of $R$ then it is a left(right) $R$-module. So, every  finitely  generated left(right) ideals of $R$ are finitely generated submodule.
  \item[5.] A submodule of left $R$-module is cyclic iff it is prinicipal ideal of $_RR$.
\end{description}
\bigskip
\begin{defn}
  Let $M$ be a left $R$-module and $\left\lbrace N_\alpha\right\rbrace_{\alpha \in \Omega }$ \marginnote{where $\Omega$ is indexing set. } be family of submodules of $M$ then sum $\sum\limits_{\alpha \in \Omega } N_\alpha $ is defined to be a submodule of $M$ generated by $\bigcup\limits_{\alpha \in \Omega}N_\alpha $  \[\left\langle \bigcup_{\alpha \in \Omega}N_\alpha \right\rangle = \sum\limits_{\alpha \in \Omega } N_\alpha \] \bigskip
   Moreover, $\sum\limits_{\alpha \in \Omega } N_\alpha $ is smallest submodule of $M$ containing $N_\alpha $ \marginnote[-2em]{for each $\alpha \in \Omega$}
  \end{defn} \bigskip
  \begin{prop}
    Let $M$ be a left $R$-module and $\left\lbrace N_\alpha\right\rbrace_{\alpha \in \Omega }$ \marginnote{where $\Omega$ is indexing set. } be family of submodules of $M$ then sum \[\sum\limits_{\alpha \in \Omega } N_\alpha =\left\lbrace \sum\limits_{\alpha \in \Omega } x_\alpha \middle\vert\, x_\alpha \in N_\alpha \quad ,\  x_\alpha = 0 \; \textmd{for almost all} \; \alpha  \right\rbrace \] \marginnote[-2em]{for each $\alpha \in \Omega$}
    \begin{proof}
      Let \[P = \left\lbrace \sum\limits_{\alpha \in \Omega } x_\alpha \middle\vert\, x_\alpha \in N_\alpha \quad , \ x_\alpha = 0 \; \textmd{for almost all} \; \alpha\right\rbrace \]
      We need to show that $P$ is smallest submodule of $M$ containing each $N_\alpha$
      \newline  \bigskip
      \textbf{Claim 1 :}.  \qquad $P$ is submodule of $M$  \newline Clearly, $P$ is non-emptpy. Taking $x_\alpha = 0$ for each $\alpha \in \Omega$, we have \[\Rightarrow0 \in P\] Also, for a fixed but arbitirary $\alpha \in \Omega $ \newline Let $x\in N_\alpha$ and  choose \[ x_i = \begin{cases}
                                 x, & \mbox{if } i =\alpha \\
                                 0, & \mbox{otherwise}.
                               \end{cases}\]
            so $x =\sum x_i \in P$  we have $N_\alpha \subseteq P \qquad \forall \alpha \in \Omega$ \marginnote{$\because \ \alpha$ was arbitrary chosen}
      \newline  \bigskip
      Now , we'll show that $P$ is submodule of $M$ using submodule criterion.\newline
      \begin{description}
        \item[\textbf{$P$ is closed under addition and scaler multiplication}]   Let $u\ , \ v$ be two elements of $P$
        \[u = \sum_{\alpha} x_\alpha \] \marginnote[-2.5em]{where $x_\alpha \in N_\alpha$ and \newline $x_\alpha = 0$ for almost all $\alpha$}
        \begin{align*} v &= \sum_{\beta} y_\beta \mathnote{where $y_\beta \in N_\beta$ and \newline $y_\beta = 0$ for almost all $\beta$}
       \end{align*} \newpage
        Let $\Omega_1$ ,  $\Omega_2$ be finite subsets of  $\Omega$ for which  $x_\alpha$ and $y_\beta$ are non-zero respectively.
        \[x_\alpha = \begin{cases}
           \text{non-zero}, & \mbox{if } \alpha \in \Omega_1 \\
          0, & \mbox{otherwise}.
        \end{cases} \qquad , \qquad y_\beta = \begin{cases}
          \text{non-zero}, & \mbox{if } \beta \in \Omega_2 \\
          0, & \mbox{otherwise}.
        \end{cases} \]
        Also for any arbitirary scaler $c \in R$ , define
        \[z_r = \begin{cases}
                  x_r, & \mbox{if } r \in \Omega_1 \\
                  c y_r, & \mbox{if } r \in \Omega_2.

                   \end{cases} \qquad \Rightarrow z_r = x_r + c y_r \qquad \forall r\ \in \Omega_1 \cap \Omega_2\]
        Now ,  \begin{align*}u + c v & = \sum_{\alpha} x_\alpha + c \sum_{\beta} y_\beta \\ & = \sum_{\alpha \in \Omega_1} x_\alpha + c \sum_{\beta \in \Omega_2} y_\beta \\ & = \sum_{\alpha \in \Omega_1} x_\alpha +  \sum_{\beta \in \Omega_2} c y_\beta \\& = \sum_{r \in \Omega_1} x_r +  \sum_{r \in \Omega_2} c y_r + \sum_{r \in \Omega_1 \cap \Omega_2} (x_r + c y_r)      \\ \text{Thus , We have}\\ u+cv &= \sum_{r \in \Omega_1 \cup \Omega_2} z_r   \in P \mathnote{$z_r = 0 \quad \text{for almost all } r \newline \because \Omega_1 \cup \Omega_2 \text{ is also finite.}$}
        \end{align*}
         \end{description}
      \textbf{Claim 2 :}.  \qquad $P$ is smalleat submodule of $M$ containing each $N_\alpha$. \newline \bigskip
      Let $N$ be any submodule of $M$ containing each $N_\alpha$ \newline $\therefore \ N$ is closed under addition   so $N$ contains all finite sum of the form $\sum\limits_{\alpha}x_\alpha $  where $ x_\alpha \in N_\alpha \quad \& \quad x_\alpha = 0 \ ,\text{for almost all }\alpha $ \newline  It follows that  $P \subseteq N$     \marginnote{$\because \ N$ was chosen arbitirary}
      \newline \bigskip Thus, $P$ is smallest submodule of $M$ containing each $N_\alpha$ \end{proof}   \end{prop}
      \bigskip
      \begin{defn}[\textbf{Maximal Submodule}]
      Let $N$ be a submodule of left $R$-module $M$ then $N$ is said to be a maximal submodule of $M$ if there does not exist any proper submodule of $M$.\newline\bigskip
      In other words , for any sumodule $K$ of $M$ satisfing
      $N\subseteq K \subseteq M$ \newline we must have \[\text{either  } N = K \quad \text{or  } K = M  \] for $N$ to be a maximal submodule of $M$.
      \end{defn}
                   \bigskip

      \begin{thm}
        Let $M$ be a finitely generated left $R$-module then every proper submodule of $M$ is contained in maximal submodule of $M$.\newline In particular , if $M$ is non-trivial then $M$ contains a maximal submodule.
      \end{thm}
       \begin{proof}
         Left to Reader.
       \end{proof}
       \bigskip
       \begin{remark}
         $\mathbb{Q} $ is not finitely generated $\mathbb{Z}$-module.
       \end{remark}
       \begin{description}
         \item[Verification:] Let  $\mathbb{Q} $ is finitely generated over $\mathbb{Z}$ by
         \[ \left\lbrace\frac{p_i}{q_i} \ \middle\vert\ p_i , q_i \in \mathbb{Z} \ , \ q_i \neq 0 \quad \forall \ i \in J_n\right\rbrace \]      Without loss of generality , Assume that \[q_1 , q_2 , q_3 , \cdots , q_n > 0\] then we can always choose an integer (say $k$) such that
         \[k = q_1  q_2  q_3  \cdots  q_n > 0 \]
         <incomlete>
       \end{description} 

\backmatter
% Encoding: UTF-8

% bibliography, glossary and index would go here.
\end{document} 