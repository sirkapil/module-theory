\documentclass{tufte-book}
\hypersetup{colorlinks}

\title{Module Theory\thanks{Thanks to Prof. Anuj Bishnoi  for his inspiration.}}
\author[Kapil Chaudhary]{Kapil Chaudhary}
\publisher{University of Delhi}

\usepackage{amsmath}
\usepackage{amsthm}
\usepackage{amsfonts}
\usepackage{amssymb}
\usepackage{lipsum}
\usepackage{hyperref}
%%
% For nicely typeset tabular material
\usepackage{booktabs}
\setcounter{section}{1}
\setcounter{chapter}{1}
\renewcommand{\theequation}{\thesection.\arabic{equation}}
\renewcommand{\thesection}{\arabic{section}}
\renewcommand{\thesubsection}{(\alph{subsection})}
%%
\renewcommand\qedsymbol{$\blacksquare$}
\theoremstyle{theorem}
\newtheorem{thm}{Theorem}[chapter]
\newtheorem{definition}{Definition}[chapter]
\newtheorem{lemma}[thm]{Lemma}
\newtheorem{prop}[thm]{Proposition}
\newtheorem{cor}[thm]{Corollary}
\newtheorem{defn}[definition]{Definition}
\newtheorem{examp}[chapter]{Example}
\newtheorem{conj}[thm]{Conjecture}
\newtheorem{remark}[section]{Remark:}
\numberwithin{equation}{chapter}
\DeclareRobustCommand{\eqref}[1]{\eqrefaux#1\eqrefaux}
\def\eqrefaux eq#1\eqrefaux{\textup{(#1)}}
% For graphics / images
\usepackage{graphicx}
\setkeys{Gin}{width=\linewidth,totalheight=\textheight,keepaspectratio}
\graphicspath{{graphics/}}

% The fancyvrb package lets us customize the formatting of verbatim
% environments.  We use a slightly smaller font.
\usepackage{fancyvrb}
\fvset{fontsize=\normalsize}

%%
% Prints argument within hanging parentheses (i.e., parentheses that take
% up no horizontal space).  Useful in tabular environments.
\newcommand{\hangp}[1]{\makebox[0pt][r]{(}#1\makebox[0pt][l]{)}}

%%
% Prints an asterisk that takes up no horizontal space.
% Useful in tabular environments.
\newcommand{\hangstar}{\makebox[0pt][l]{*}}

%%
% Prints a trailing space in a smart way.
\usepackage{xspace}
\usepackage{cancel}
%%
% Some shortcuts for Tufte's book titles.  The lowercase commands will
% produce the initials of the book title in italics.  The all-caps commands
% will print out the full title of the book in italics.
\newcommand{\vdqi}{\textit{VDQI}\xspace}
\newcommand{\ei}{\textit{EI}\xspace}
\newcommand{\ve}{\textit{VE}\xspace}
\newcommand{\be}{\textit{BE}\xspace}
\newcommand{\VDQI}{\textit{The Visual Display of Quantitative Information}\xspace}
\newcommand{\EI}{\textit{Envisioning Information}\xspace}
\newcommand{\VE}{\textit{Visual Explanations}\xspace}
\newcommand{\BE}{\textit{Beautiful Evidence}\xspace}

\newcommand{\TL}{Tufte-\LaTeX\xspace}
\newcommand{\bigslant}[2]{{\raisebox{.1em}{$#1$}\left/\raisebox{-.2em}{$#2$}\right.}}
% Prints the month name (e.g., January) and the year (e.g., 2008)
\newcommand{\monthyear}{%
	\ifcase\month\or January\or February\or March\or April\or May\or June\or
	July\or August\or September\or October\or November\or
	December\fi\space\number\year
}


% Prints an epigraph and speaker in sans serif, all-caps type.
\newcommand{\openepigraph}[2]{%
	%\sffamily\fontsize{14}{16}\selectfont
	\begin{fullwidth}
		\sffamily\large
		\begin{doublespace}
			\noindent\allcaps{#1}\\% epigraph
			\noindent\allcaps{#2}% author
		\end{doublespace}
	\end{fullwidth}
}

% Inserts a blank page
\newcommand{\blankpage}{\newpage\hbox{}\thispagestyle{empty}\newpage}

\usepackage{units}
\def\mathnote#1{%
  \tag*{\rlap{\hspace\marginparsep\smash{\parbox[t]{\marginparwidth}{%
  \footnotesize#1}}}}
}% Typesets the font size, leading, and measure in the form of 10/12x26 pc.
\newcommand{\measure}[3]{#1/#2$\times$\unit[#3]{pc}}

% Macros for typesetting the documentation
\newcommand{\hlred}[1]{\textcolor{Maroon}{#1}}% prints in red
\newcommand{\hangleft}[1]{\makebox[0pt][r]{#1}}
\newcommand{\hairsp}{\hspace{1pt}}% hair space
\newcommand{\hquad}{\hskip0.5em\relax}% half quad space
\newcommand{\TODO}{\textcolor{red}{\bf TODO!}\xspace}
\newcommand{\na}{\quad--}% used in tables for N/A cells
\providecommand{\XeLaTeX}{X\lower.5ex\hbox{\kern-0.15em\reflectbox{E}}\kern-0.1em\LaTeX}
\newcommand{\tXeLaTeX}{\XeLaTeX\index{XeLaTeX@\protect\XeLaTeX}}
% \index{\texttt{\textbackslash xyz}@\hangleft{\texttt{\textbackslash}}\texttt{xyz}}
\newcommand{\tuftebs}{\symbol{'134}}% a backslash in tt type in OT1/T1
\newcommand{\doccmdnoindex}[2][]{\texttt{\tuftebs#2}}% command name -- adds backslash automatically (and doesn't add cmd to the index)
\newcommand{\doccmddef}[2][]{%
	\hlred{\texttt{\tuftebs#2}}\label{cmd:#2}%
	\ifthenelse{\isempty{#1}}%
	{% add the command to the index
		\index{#2 command@\protect\hangleft{\texttt{\tuftebs}}\texttt{#2}}% command name
	}%
	{% add the command and package to the index
		\index{#2 command@\protect\hangleft{\texttt{\tuftebs}}\texttt{#2} (\texttt{#1} package)}% command name
		\index{#1 package@\texttt{#1} package}\index{packages!#1@\texttt{#1}}% package name
	}%
}% command name -- adds backslash automatically
\newcommand{\doccmd}[2][]{%
	\texttt{\tuftebs#2}%
	\ifthenelse{\isempty{#1}}%
	{% add the command to the index
		\index{#2 command@\protect\hangleft{\texttt{\tuftebs}}\texttt{#2}}% command name
	}%
	{% add the command and package to the index
		\index{#2 command@\protect\hangleft{\texttt{\tuftebs}}\texttt{#2} (\texttt{#1} package)}% command name
		\index{#1 package@\texttt{#1} package}\index{packages!#1@\texttt{#1}}% package name
	}%
}% command name -- adds backslash automatically
\newcommand{\docopt}[1]{\ensuremath{\langle}\textrm{\textit{#1}}\ensuremath{\rangle}}% optional command argument
\newcommand{\docarg}[1]{\textrm{\textit{#1}}}% (required) command argument
\newenvironment{docspec}{\begin{quotation}\ttfamily\parskip0pt\parindent0pt\ignorespaces}{\end{quotation}}% command specification environment
\newcommand{\docenv}[1]{\texttt{#1}\index{#1 environment@\texttt{#1} environment}\index{environments!#1@\texttt{#1}}}% environment name
\newcommand{\docenvdef}[1]{\hlred{\texttt{#1}}\label{env:#1}\index{#1 environment@\texttt{#1} environment}\index{environments!#1@\texttt{#1}}}% environment name
\newcommand{\docpkg}[1]{\texttt{#1}\index{#1 package@\texttt{#1} package}\index{packages!#1@\texttt{#1}}}% package name
\newcommand{\doccls}[1]{\texttt{#1}}% document class name
\newcommand{\docclsopt}[1]{\texttt{#1}\index{#1 class option@\texttt{#1} class option}\index{class options!#1@\texttt{#1}}}% document class option name
\newcommand{\docclsoptdef}[1]{\hlred{\texttt{#1}}\label{clsopt:#1}\index{#1 class option@\texttt{#1} class option}\index{class options!#1@\texttt{#1}}}% document class option name defined
\newcommand{\docmsg}[2]{\bigskip\begin{fullwidth}\noindent\ttfamily#1\end{fullwidth}\medskip\par\noindent#2}
\newcommand{\docfilehook}[2]{\texttt{#1}\index{file hooks!#2}\index{#1@\texttt{#1}}}
\newcommand{\doccounter}[1]{\texttt{#1}\index{#1 counter@\texttt{#1} counter}}

% Generates the index
\usepackage{makeidx}
\makeindex


\begin{document}
\frontmatter

\maketitle
\newpage
\begin{fullwidth}
	~\vfill
	\thispagestyle{empty}
	\setlength{\parindent}{0pt}
	\setlength{\parskip}{\baselineskip}
	Copyright \copyright\ \the\year\ , All rights are reserved. \newline \thanklessauthor
		
	\par\smallcaps{\thanklesspublisher}
	
	\par\texttt{\url{https://contact.sirkapil.me/}}
	\par Licensed under the Apache License, Version 2.0 (the ``License''); you may not
	use this file except in compliance with the License. You may obtain a copy
	of the License at \url{http://www.apache.org/licenses/LICENSE-2.0}. Unless
	required by applicable law or agreed to in writing, software distributed
	under the License is distributed on an \smallcaps{``AS IS'' BASIS, WITHOUT
		WARRANTIES OR CONDITIONS OF ANY KIND}, either express or implied. See the
	License for the specific language governing permissions and limitations
	under the License.\index{license}
	\par\textit{First Print, \monthyear}
\end{fullwidth}

\tableofcontents
\cleardoublepage
\begin{fullwidth}
~\vfill
\begin{doublespace}
	\noindent\fontsize{18}{22}\selectfont\itshape
	\nohyphenation
   There are few persons without whom this was impossible.\newline
   They deserve credit for it so i would love to thank them.\newline
\vfill
	Special Thanks to :\newline
        \begin{center}
        -  Dr. Anuj Bishnoi (Subject \ Teacher)\newline
		-	Edward Tufte (\LaTeX \  Tufte  Templete)
		\end{center}
\end{doublespace}
\vfill
\vfill
\end{fullwidth}
\newpage
\begin{fullwidth}
~\vfill
\vfill
\begin{doublespace}
	\noindent\fontsize{18}{22}\selectfont\itshape
	\nohyphenation
	\centering
	Dedicated to my family and my best friend\newline\texttt{Neeraj K. Gaud}
\end{doublespace}
\vfill
\vfill
\end{fullwidth}
\newpage

\chapter*{Introduction}
\section{Warning :}
This is my first document created using latex so it may be possible that there are several errors. if you notice any  error then you can report it here.\newline
\ \url{https://github.com/sirkapil/module-theory/issues/new}\footnote{(may require a github account)}
\section{About :}
This sample book discusses the course "Module Theory" being taught to Post-Graduate (M.Sc. Mathematics) students in Department of Mathematics under University of Delhi, Delhi.\newline \bigskip
All my \ \LaTeX \ documents are free and open-source. Each document is hosted in a github repository and can be found pinned here.\newline \url{https://github.com/sirkapil} 

\section{Contribution :}
If you find my work useful and want to contribute then you are welcome by heart.\newline Any suitable changes to document repository through pull requests are highly appreciated. You can create a new pull request here. Be sure to read \textit{contribution file} in root/.github folder of repository before creating any pull-request. \newline\url{https://github.com/sirkapil/module-theory/compare}\newline\bigskip If you don't have a github account or facing difficulty in creating a pull-request , then feel free to drop down a message here about that you are interested in contribution of this project.\newline \url{https://cont.netlify.com}\newline\url{https://twitter.com/kapil_rc}

%%
% Start the main matter (normal chapters)
\mainmatter
\chapter{Introduction to Modules}
\section{\texttt{Defination of Module}}
\subsection{Left Module:}\label{sec:left-module}\index{Left Module}
 Let $ R $ be a ring with identity and $ M $ be an abelian group with addition. We say $ M $ is a left $R-$module if there exists a mapping\footnote{often called as scaler multiplication.}
\begin{equation*}
R \times M \rightarrow M
\end{equation*}	
defined by 
\begin{equation*}
(a \ , \ x) \rightarrow ax  
\end{equation*}
for each $a$ $\in$ $R$ and $x$ $\in$ $M$ satisfying following properties :
\marginnote{$\forall$ $ a\ , \ b \in R$ and $x \ , y \ \in M$  }
\begin{eqnarray}
(a+b) x &=& ax + bx\label{1} \\ a(x+y) &=& ax + ay\label{2} \\ (ab)x &=& a(bx)\label{3} \\ 1x &=& x\label{4}
\end{eqnarray}
and denoted by $_{R}M$
\subsection{Right Module:}\label{sec:right-module}\index{Right Module}
Let $ R $ be a ring with identity and $ M $ be an abelian group with addition. We say $ M $ is a right $R-$module if there exists a mapping
\begin{equation*}
M \times R \rightarrow M
\end{equation*}	
defined by 
\begin{equation*}
(x \ ,\ a) \rightarrow xa  
\end{equation*}
for each $a$ $\in$ $R$ and $x$ $\in$ $M$ satisfying following properties\marginnote{$\forall$ $ a\ , \ b \in R$ and $x \ , y \ \in M$  } :
\begin{eqnarray}
x (a+b)  &=& xa + xb\label{5} \\ (x+y) a &=& xa + ya\label{6} \\ x (ab) &=& (xa)b\label{7} \\ x1 &=& x\label{8}
\end{eqnarray}
and denoted by $M_{R}$

\subsection{Examples :}
\begin{enumerate}
\item Let $V$ be a vector space over a field $F$ then $V$ is a left as well as right $F-$Module.\newline \bigskip
\item Let $G$ be any abelian group under addition , then $G$ is a $\mathbb{Z-}$Module where $\mathbb{Z}$ is set of integers.\newline \bigskip
\item Let $R$ be ring and $M= R[x]$ \marginnote{Suppose ring $R$ is a field then $R-$Module $R[x]$ is a vector space over field $R$.} where $R[x]$ is a group of all polynomials with coefficents in $R$ then $M$ is a left as well as a right $R-$Module with scaler multiplication being usual multiplication.\newline \bigskip
\item Let $M$ be collection of all $m \times n $ matrices over ring $R$ , then $M$ is left $R-$Module where scaler multiplication being usual multiplication of a scaler to a matrix. \newline
\bigskip
In particular, if $M$ is a set of $1 \times n$ matrices over $R$ or $M = R^n$(set of $n-$tuples) then $R^n$ is a left $R-$module.
\end{enumerate}
\bigskip
\begin{remark}
	Let $R$ be a commutative ring then every left $R-$module can be transformed to right $R-$module and vice-versa.
	\end{remark}
\begin{proof} Let $M$ be left $R-$module and $R$ be a commutative ring. \newline
	so, $\exists$ a mapping
\begin{equation*}
R \times M \rightarrow M
\end{equation*}	
defined by 
\begin{equation*}
(a \ , \ x) \rightarrow ax  
\end{equation*}
for each $a$ $\in$ $R$ and $x$ $\in$ $M$ satisfying following properties :
\marginnote{$\forall$ $ a\ , \ b \in R$ and $x \ , y \ \in M$  }
\begin{eqnarray*}
(a+b) x &=& ax + bx \\ a(x+y) &=& ax + ay \\ (ab)x &=& a(bx) \\ 1x &=& x
\end{eqnarray*}
$\because \ R$ is a commutative ring.\newline
Now, Define an another mapping
 \begin{equation*}
 M \times R \rightarrow M
 \end{equation*}	
 defined by 
 \begin{equation*}
 (x \ ,\ a) \rightarrow x*a \ = \ ax  
 \end{equation*}
To check $M$ is a right $R-$Module , we need to verify properties number \eqref{5}-\eqref{8}
 \begin{enumerate}
 	\item 
 	\begin{align*}
 		x*(a+b) &= (a+b)x \\ &= ax + bx \\ &= (x*a) + (x*b)
 	\end{align*}
 	\item \begin{eqnarray*}
 		(x+y)*a &=& a(x+y) \\ &=& ax + ay \\ &=& (x*a) + (y*a)
 	      \end{eqnarray*}
       \item \begin{eqnarray*}
       x*(ab) &=& (ab)x \\ &=& (ba)x \\ &=& b(ax) \\ &=& (ax)*b
       \end{eqnarray*} 
   \item \begin{eqnarray*}
   x*1 &= 1x \\  &= x
   \end{eqnarray*}
Thus, $_{R}M$ is transformed to $M_R$.\newline
Similarly, Converse statement can be verified.
\end{enumerate}
\end{proof} \bigskip
\begin{remark}
	Let $S$ be a subring of ring $R$ then $_{S}M$ exists \marginnote{by existance means $M$ is a valid left module over mentioned ring or subring. i.e. satisfying those four properties.}  only if $_{R}M$ exists.
\end{remark} \bigskip 
\begin{remark}
	Same Abelian group\marginnote{For Instance,  The field $\mathbb{R}$ is $\mathbb{R}-$module,$\mathbb{Q}-$module and $\mathbb{Z}-$module.} can have the structure of a Module for a number of different rings. 
\end{remark} \bigskip 
\begin{remark}
Let $I$ be left ideal of $R$ then quotient ring $\bigslant{R}{I}$ is a left $R$-module.\marginnote{Here scaler multiplication is \newline \begin{equation*} 
	R \times \bigslant{R}{I} \rightarrowtail \bigslant{R}{I} 
	\end{equation*}
defined as 
\begin{equation*}
(a \ , \ x+I) \rightarrowtail ax+I
\end{equation*}

$\forall \ a \in R$ and $\forall \ x+I \in \bigslant{R}{I}$} 

\end{remark} 
\begin{proof}[verification:]
	`left to reader'\newline \bigskip
	\textbf{Hint:} you need to verify those four properties: \eqref{1}-\eqref{4}
\end{proof}
\bigskip
\begin{theorem}{(Elementry Properties:)}\newline
	Let $M$ be a left $R$-module . Suppose $0_m \ \text{and} \ 0_r$ denotes additive identities of $M$ and $R$ respectively. Then,  for each $x \ \in M$ and $r \ \in R$ \newline
\begin{description}
\item (i)
	\begin{equation*}
	0_m = 0_r\ x = r\ 0_m
	\end{equation*}
\item (ii)
	\begin{equation*}
	r(-x) =  (-r)x = -rx 
	\end{equation*}
\end{description}
\end{theorem}
\begin{proof}
	\begin{description}
		\item[(i)]
As $0_m$ is the additive identity of $M$. so, $0_m = 0_m + 0_m$\newline
Consider \marginnote{$ \because \ (r\ ,\ 0_m)\rightarrowtail r\ 0_m \in M$\newline so, $r\ 0_m = r\ 0_m + 0_m$}
 $ r(0_m + 0_m) = r\ 0_m  = r\ 0_m +0_m $\newline
but, $r(0_m + 0_m) = (r\ 0_m) +(r\ 0_m)$ \marginnote{
	$ \because \ M$ is a left $R$-module.\newline (using distribuitive property)}\newline
so, we have \begin{equation*}
r\ 0_m +r\ 0_m = r\ 0_m + 0_m
\end{equation*}
as ($M,+$) is an abelian group so left and right cancellation law holds.
\begin{eqnarray*}
\cancel{r\ 0_m} +r\ 0_m &=& \cancel{r\ 0_m} + 0_m \\ r\ 0_m &=& 0_m
\end{eqnarray*}
a similiar argument can be used to prove $0_m = 0_r\ x$.
\item[(ii)]
as $M$ is a left $R$-module so $(r \ , x )\ \rightarrowtail rx \in M$ \newline
Now, Consider $(-r)x + rx$
\begin{eqnarray*}
\text{using distribuitive law} \\ (-r)x + rx &=& (-r + r)x \\  &=& 0_r \ x \\ &=& 0_m
\end{eqnarray*}
i.e. $(-r)x$ is additive inverse of  $(rx)$ but additive inverse of $(rx)$ is $-rx$ and it is unique for an abelian group($M$ here)
\begin{equation*}
\therefore \ (-r)x = -rx
\end{equation*}
a similar argument can be used to prove that $r(-x)  = -r x$.
\end{description}
\end{proof}
\bigskip
\begin{definition}[Ring Homomorphism]
	Let $R$ and $S$ be two rings with identities $1_r \ , 1_s$ respectively then a map(say $f$)
	\begin{equation*}
	f : R \rightarrow S
	\end{equation*}
	is said to be a ring homomorphism (often said ring homo) or ring linear map if for every $a \ ,\ b \in R$ following properties holds
	\begin{description}
		\item[(i)]\begin{equation*}
		(a+b)f = (a)f +(b)f
		\end{equation*}
		\item[(ii)]\begin{equation*}
		(ab)f = (a)f \ (b)f
		\end{equation*}
		\item[(iii)]\begin{equation*}
		(1_r)f = 1_s
		\end{equation*}
	\end{description}
\end{definition}


\chapter{ Stay Tuned for next chapters!}
\backmatter
% Encoding: UTF-8

% bibliography, glossary and index would go here.
\end{document} 